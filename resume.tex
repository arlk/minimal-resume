%%%%%%%%%%%%%%%%%%%%%%%%%%%%%%%%%%%%%%%%%%%%%%%%%%%%%%%%%%%%%%%%%%%%%%%%
%%%%%%%%%%%%%%%%%%%%%% Simple LaTeX CV Template %%%%%%%%%%%%%%%%%%%%%%%%
% _author_: Arun Lakshmanan
% _email_: lakshma2@illinois.edu
% modified from the Simple LaTeX CV Template (github.com/kscottz)
%%%%%%%%%%%%%%%%%%%%%%%%%%%%%%%%%%%%%%%%%%%%%%%%%%%%%%%%%%%%%%%%%%%%%%%%

%%%%%%%%%%%%%%%%%%%%%%%%%%%% Document Setup %%%%%%%%%%%%%%%%%%%%%%%%%%%%

\documentclass[11pt]{article}

% Input details here:
\newcommand{\authorname}{Arun Lakshmanan}
\newcommand{\addr}{Urbana IL, 61801}
\newcommand{\email}{\href{mailto:lakshma2@illinois.edu}{lakshma2@illinois.edu}}
\newcommand{\github}{\href{http://github.com/arlk}{github.com/arlk}}
\newcommand{\website}{\href{www.arunl.com}{arunl.com}}

% This is a helpful package that puts math inside length specifications
\usepackage{calc}
\usepackage{textcomp}

% Layout: Puts the section titles on left side of page
\reversemarginpar

% Setup colors
\usepackage{color}
\definecolor{darkblue}{rgb}{0.0,0.0,0.3}
\definecolor{lightgray}{rgb}{0.5,0.5,0.5}

%% Use these lines for letter-sized paper
\usepackage[paper=letterpaper,
            %includefoot, % Uncomment to put page number above margin
            marginparwidth=1.00in,     % Length of section titles
            marginparsep=.05in,       % Space between titles and text
            margin=0.75in,
            includemp]{geometry}

%% More layout: Get rid of indenting throughout entire document
\setlength{\parindent}{0in}

%% This gives us fun enumeration environments. compactitem will be nice.
\usepackage{paralist}

%% Last updated footer
\usepackage{fancyhdr}
\usepackage[en-US]{datetime2}

\fancyhf{}
\renewcommand{\headrulewidth}{0pt}
\fancyfoot[C]{\color{lightgray}\footnotesize Last updated on \today}
\pagestyle{fancy}
\pagenumbering{gobble}

% PDF bookmarks
\usepackage{hyperref}
\hypersetup{colorlinks,breaklinks,pageanchor=false,
            linkcolor=darkblue,urlcolor=darkblue,
            anchorcolor=darkblue,citecolor=darkblue}
            
% Asterisk Footnotes 
\newcommand\astfootnote[1]{%
  \begingroup
  \renewcommand\thefootnote{}\footnote{#1}%
  \addtocounter{footnote}{-1}%
  \endgroup
}

%%%%%%%%%%%%%%%%%%%%%%%% End Document Setup %%%%%%%%%%%%%%%%%%%%%%%%%%%%


%%%%%%%%%%%%%%%%%%%%%%%%%%% Helper Commands %%%%%%%%%%%%%%%%%%%%%%%%%%%%

% The title (name) with a horizontal rule under it
%
% Usage: \makeheading{name}
%
% Place at top of document. It should be the first thing.

\newcommand{\makeheading}%
         {\vspace*{-2.5em}%
          \hspace*{-\marginparsep minus \marginparwidth}%
             \begin{minipage}[t]{\textwidth+\marginparwidth+\marginparsep}%
               {\Huge \hspace{-1.20em} \authorname}%
             \end{minipage}
          \hfill%
             \begin{minipage}[t]{\textwidth}%
               \begin{flushright}%
                 \vspace{-4.0em}%
                 \email \\
                 \github \\
                 \website
               \end{flushright}
             \end{minipage}}

% The section headings
%
% Usage: \section{section name}
%
% Follow this section IMMEDIATELY with the first line of the section
% text. Do not put whitespace in between. That is, do this:
%
%       \section{My Information}
%       Here is my information.
%
% and NOT this:
%
%       \section{My Information}
%
%       Here is my information.
%
% Otherwise the top of the section header will not line up with the top
% of the section. Of course, using a single comment character (%) on
% empty lines allows for the function of the first example with the
% readability of the second example.
\renewcommand{\section}[2]%
        {\pagebreak[2]\vspace{1.3\baselineskip}%
         \phantomsection\addcontentsline{toc}{section}{#1}%
         \hspace{0in}%
         \marginpar{
         \raggedright \scshape #1}#2}

% An itemize-style list with lots of space between items
\newenvironment{outerlist}[1][\enskip\textbullet]%
        {\begin{itemize}[#1]}{\end{itemize}%
         \vspace{-.6\baselineskip}}

% An environment IDENTICAL to outerlist that has better pre-list spacing
% when used as the first thing in a \section
\newenvironment{lonelist}[1][\enskip\textbullet]%
        {\vspace{-\baselineskip}\begin{list}{#1}{%
        \setlength{\partopsep}{0pt}%
        \setlength{\topsep}{0pt}}}
        {\end{list}\vspace{-.6\baselineskip}}

% An itemize-style list with little space between items
\newenvironment{innerlist}[1][\enskip\textbullet]%
        {\begin{compactitem}[#1]}{\end{compactitem}}

% To add some paragraph space between lines.
% This also tells LaTeX to preferably break a page on one of these gaps
% if there is a needed pagebreak nearby.
\newcommand{\blankline}{\quad\pagebreak[2]}

% Software macros
\newcommand \CPP        {{C\nolinebreak[4]\hspace{-.05em}\raisebox{.4ex}{\tiny\bf ++}}}

%%%%%%%%%%%%%%%%%%%%%%%% End Helper Commands %%%%%%%%%%%%%%%%%%%%%%%%%%%

%%%%%%%%%%%%%%%%%%%%%%%%% Begin CV Document %%%%%%%%%%%%%%%%%%%%%%%%%%%%

\begin{document}

\makeheading

\section{Research Interests}
Using tools from statistical learning theory to address problems arising from model uncertainty in robotics and control theory with its relevance to motion planning; safe simultaneous learning and control for robots; collision detection methods for motion planning; safe model predictive control under uncertainty; fast planning with reduced-order models. \medskip

\section{Education}
\textbf{Ph.D. Mechanical Science and Engineering}   \hfill \textbf{Jan 2017 - }\textbf{present}\\
\href{http://naira.mechse.illinois.edu/}{Advanced Controls Research Laboratory}\\
\textit{Advisor: \href{http://naira.mechse.illinois.edu/sciencex_teams/naira-hovakimyan/}{Naira Hovakimyan}}\\
\textit{\href{http://www.mechse.illinois.edu/}
        {University of Illinois at Urbana-Champaign},
        Urbana, IL.} \bigskip

\textbf{M.S. Aerospace Engineering}   \hfill \textbf{Aug 2014 - Dec 2016}\\
\textit{\href{http://www.ae.illinois.edu/}
        {University of Illinois at Urbana-Champaign},
        Urbana, IL.}  \bigskip

\textbf{B.Tech. Mechanical Engineering}  \hfill \textbf{July 2010 - May 2014}\\
\textit{\href{http://vit.ac.in/}
        {VIT University},
        Vellore, India.} \medskip

\section{Employment}
\textbf{Facebook Reality Labs}, Redmond, WA.
\hfill \textbf{May 2018 - Aug 2018}\\
\textit{Advisors: Douglas Lanman, Nick Colonnese}\\
\textit{Research Intern} \bigskip

\textbf{Paracosm/Occipital}, Gainesville, FL.
\hfill \textbf{May 2017 - Jul 2017}\\
\textit{Mentors: Jack Morrison, Quinn Martin}\\
\textit{Robotics Perception Intern} \bigskip

\textbf{Qualcomm Research}, Philadelphia, PA.
\hfill \textbf{May 2016 - Aug 2016}\\
\textit{Advisor: Matthew Turpin}\\
\textit{Research Intern} \medskip

\section{Technical Summary}
Proficiency of programming languages (from most to least comfortable): Julia, \CPP, C, Python, Simulink. Extensive experience with robotics development on quadrotors and ground robots - typically this would involve a \CPP/C firmware stack, ROS/LCM communication middleware, and a Julia/Python/Simulink layer that handles the high-level decision making. The following are some of the open-source packages published on Github:\\\medskip

\begin{lonelist}
\item \href{https://github.com/arlk/SafeFeedbackMotionPlanning.jl}{SafeFeedbackMotionPlanning.jl}: A Julia package for designing nonlinear controllers that ensure guaranteed performance in trajectory tracking problems.
\item \href{https://github.com/arlk/CurveProximityQueries.jl}{CurveProximityQueries.jl}, \href{https://github.com/arlk/CurveProximityQueries.jl}{ConvexBodyProximityQueries.jl}: Packages that implements methods to compute proximity queries between convex bodies and/or parametric curves in 2/3D.
\item \href{https://github.com/arlk/cf-firmware}{cf-firmware}: A firmware fork of the original Crazyflie repository that additionally implements geometric control, path following for trajectories, and handles control for attached manipulators.
\end{lonelist}

\newpage

\section{Publications}
\begin{lonelist}
\item[] Aditya Gahlawat$^\star$, \textbf{Arun Lakshmanan$^\star$}, Lin Song, Andrew Patterson, Zhuohuan Wu, and Naira Hovakimyan. \\
$\mathcal{RL}_1-\mathcal{GP}$: Safe simulataneous learning and control. \textit{(submitted to the Conference on Robot Learning (CoRL) 2020)}. \bigskip

\item[] \textbf{Arun Lakshmanan$^\star$}, Aditya Gahlawat$^\star$, and Naira Hovakimyan. \\
Safe feedback motion planning: A contraction theory and $\mathcal{L}_1$-adaptive control based approach. In \textit{IEEE Conference on Decision and Control 2020}, Dec 2020.\astfootnote{$^\star$equal contribution}\\
\url{https://arxiv.org/pdf/2004.01142.pdf} \bigskip

\item[] Andrew Patterson, \textbf{Arun Lakshmanan}, and Naira Hovakimyan. \\
Intent-aware probabilistic trajectory estimation for collision prediction with uncertainty quantification. In \textit{IEEE Conference on Decision and Control 2019}, Dec 2019.\\
\url{https://ieeexplore.ieee.org/document/9029215} \bigskip

\item[] \textbf{Arun Lakshmanan}, Andrew Patterson, Venanzio Cichella, and Naira Hovakimyan.\\
Proximity queries for absolutely continuous curves. In \textit{Proceedings of Robotics: Science and Systems}, June 2019. \\
\url{http://www.roboticsproceedings.org/rss15/p42.pdf} \bigskip

\item[] Robert M Jones, Donglie Sun, Gabriel B Haberfeld, \textbf{Arun Lakshmanan}, Thiago Marinho, and Naira Hovakimyan.\\
Design and control of a small aerial manipulator for indoor environments. In \textit{AIAA Information Systems-AIAA Infotech@ Aerospace}, Jan 2017. \\
\url{https://arc.aiaa.org/doi/10.2514/6.2017-1374} \bigskip

\item[] \textbf{Arun Lakshmanan}.\\
Piecewise Bézier curve trajectory generation and control for quadrotors. 
\textit{Master's Thesis, University of Illinois at Urbana-Champaign}, Dec 2016. \\
\url{https://www.ideals.illinois.edu/handle/2142/95352} \bigskip

\item[] Thiago Marinho, Christopher Widdowson, Amy Oetting, \textbf{Arun Lakshmanan}, Hang Cui, Naira Hovakimyan, Ranxiao Frances Wang, Alex Kirlik, Amy Laviers, and Dušan Stipanović.\\
Carebots: Prolonged elderly independence using small mobile robots. In \textit{Mechanical Engineering, ASME}, Sep 2016. \\
\url{https://asmedigitalcollection.asme.org/memagazineselect/article-pdf/138/09/S8/6359956/me-2016-sep5.pdf} \bigskip

\item[] Thiago Marinho, \textbf{Arun Lakshmanan}, Venanzio Cichella, Christopher Widdowson, Hang Cui, Robert M Jones, Bentic Sebastian, and Camille Goudeseune.\\
VR study of human-multicopter interaction in a residential setting. In \textit{2016 IEEE Virtual Reality (VR)}, Mar 2016. \\
\url{https://ieeexplore.ieee.org/document/7504790} 

\end{lonelist}



\end{document}


%%%%%%%%%%%%%%%%%%%%%%%%%% End CV Document %%%%%%%%%%%%%%%%%%%%%%%%%%%%%
